% ------------------------------------------------
%-------------------------------------------------
% Created by Ryan Sharp u1090072
% CSC3600 - Final Report
% Due 17th of October 2019
% ------------------------------------------------
% ------------------------------------------------

% ----- Include ----- %
\documentclass[11pt]{article}
\usepackage{graphicx}
\usepackage{amssymb, amsmath}
\usepackage{multirow}
\usepackage{graphicx, color}
\usepackage{wrapfig}
\usepackage{float}

\setlength{\parindent}{0pt}

\begin{document}

% ----- Start Title Page ----- %

    \begin{titlepage}
        \centering
        \vspace {1cm}
        \huge{\textbf{CSC3600 Final Report}} \\ [0.1cm]
        \begin{figure}[ht!]
            \centering
            \def\svgwidth{0.5\columnwidth}
            \includegraphics[scale={0.20}]{USQ.jpg}
        \end{figure}
        \vspace {0.5cm}
        \Large{By} \\
        \Large{\textbf{USQ Learning Emporium}} \\
        \Large{Richard Dobson, Jesse Hare, James McKeown, Vincent Roberts, Ryan Sharp} \\[0.50cm]
        \Large{Examiner} \\
        \Large{\textbf{Dr. Xiaohui Tao}} \\
        \Large{\textit{Senior Lecturer (Computing) – School of Sciences -
PhD QUT}} \\[0.50cm]
        \Large{Project Supervisor} \\
        \Large{\textbf{Assoc. Prof. Stijn Dekeyser}} \\
        \Large{\textit{Associate Professor (Computing) - School of Sciences - PhD Antwerp}} \\[0.50cm]
        \Large{School of Management and Enterprise} \\[0.50cm]

        \Large{Due: Thursday 17th of October 2019}
    \end{titlepage}


\newpage

% ----- End Title Page ----- %



% ----- start ToC ----- %

\tableofcontents
\newpage

% ----- start ToC ----- %





\setcounter{secnumdepth}{0}
% ----- Section 1 ----- %

\section{1. Executive Summary}
This report provides an analysis and evaluation of the development stages completed for the harvest metadata.


% ----- Section 2 ----- %

\section{2. Methodology}
Lorem Ipsum

\subsection{2.1 Methodology Statement}
Lorem Ipsum

\subsection{2.2 Justifications}
Lorem Ipsum

\subsection{2.3 Discussions}
Lorem Ipsum


% ----- Section 3 ----- %

\section{3. Project Process}

\subsection{3.1 Team Organisation}
There were a total of 5 members in the USQ Learning Emporium group. As this project was initially designed for teams of 3, the group needed to strategise ways in which to split the work up. The first solution was to divide the main project task into 2 main groups; a back-end team (3 members) and a front-end team (2 members). Within these groups each member had different roles.

\subsection{3.2 Team Structure and Roles}
As stated above, the group was divided into 2 main groups;  a back-end team (3 members) and a front-end team (2 members). The back-end team consisted of: James McKeown (Team Leader), Richard Dobson (Programmer) and Vincent Roberts (Programmer) **needs to be fixed. The front-end team consisted of: Jesse Hare (Team Leader), Ryan Sharp (Programmer). \\

James McKeown and Jesse Hare both lead their respective teams. This allowed for clear and concise instructions (**expand on this). Both of these members took on the most technical aspects of the project, especially in regards to researching solutions. Other members then utilised this research to program their respective tasks within the project. \\


\subsection{3.3 Communication and Meetings}
\subsubsection{3.3.1 Effectiveness of Communication}
Communication between members was very quick and effective. The team decided to use a private Facebook Messenger chat as the primary point of contact as each member regularly used this application. This made it easier to notify team members of urgent matters. In addition, the team decided to use a Slack channel for more technical and project specific subjects. Slack was new to all members and was therefore not utilised to it's full potential. \\

The main problem faced with the Facebook Messenger chat, was that the information wasn't organised properly. It would have been better to use Slack as the primary point of contact as it was more suited to this type of project; especially because we had 5 team members and 2 sub teams. In future projects organisation of information needs to be a higher priority.

\subsubsection{3.3.2 Team Meetings}
Team meetings were run through the Zoom video conferencing application. At least once a week the whole team would meet to discuss the progress of the project. Meetings for the back end and front end team were often held more than once a week to discuss new ideas for their respective tasks. These meetings were highly effective as each member could screen share what they were working on and clearly communicate that with the team. \\

These meetings could be improved by having a set agenda before each meeting. Occasionally the meetings started slow as we all had to organise what we were going to discuss. It would also be ideal to have one member as the designated scribe to take notes for each meeting and relay this back to the team. 


\subsection{3.4 Documentation}
Lorem Ipsum

\subsection{3.5 Process}
Lorem Ipsum


% ----- Section 4 ----- %

\section{4. Project Report}
Lorem Ipsum

\subsection{4.1 Project Outcome}
Lorem Ipsum

\subsection{4.2 Cost of the Project}
Lorem Ipsum


% ----- Section 5 ----- %

\section{5. Professionalism and Professional Ethics}

\subsection{5.1 Professionalism}
Lorem Ipsum

\subsection{5.2 Professional Ethics}
Lorem Ipsum


% ----- Section 6 ----- %

\section{6. Contribution Distribution}


% ----- Section 7 ----- %

\section{7. Conclusions}



% ----- Section: Reference List ----- %

\section{Reference List}


% ----- Section: Appendix ----- %

\section{Appendix}



\end{document}

